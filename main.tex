% Metódy inžinierskej práce

\documentclass[10pt,twoside,slovak,a4paper]{article}

\usepackage[slovak]{babel}
\usepackage[IL2]{fontenc} % lepšia sadzba písmena Ľ než v T1
\usepackage[utf8]{inputenc}
\usepackage{graphicx}
\usepackage{url} % príkaz \url na formátovanie URL
\usepackage{hyperref} % odkazy v texte budú aktívne (pri niektorých triedach dokumentov spôsobuje posun textu)

\usepackage{cite}

\usepackage{multirow}
\usepackage[table,xcdraw]{xcolor}
\usepackage{colortbl}
\usepackage{array}
\usepackage{caption}

\usepackage{geometry}





\title{Výber zaujímavých ľudí a obsahu pre používateľa na sociálnych sieťach\thanks{Semestrálny projekt v predmete Metódy inžinierskej práce, ak. rok 2024/25, vedenie: Richard Marko}} % meno a priezvisko vyučujúceho na cvičeniach

\author{Kostiantyn Borysiuk\\[2pt]
	{\small Slovenská technická univerzita v Bratislave}\\
	{\small Fakulta informatiky a informačných technológií}\\
	{\small \texttt{xborysiuk@stuba.sk}}
	}

\date{\small 13. decembra 2024} % upravte



\begin{document}

\maketitle

\begin{abstract}
V modernom svete takmer každý človek, ktorý má prístup na internet, používa jednu alebo druhú sociálnu sieť, vďaka čomu je moja téma veľmi dôležitá. Hlavnou úlohou sociálnych sietí je spájať ľudí so spoločnými záujmami.
Odporúčacie systémy analyzujú akcie používateľa v sieti a na základe jeho záujmov, prezeraných materiálov, ľudí, ktorých sleduje atď., navrhujú obsah. Analyzujú sa aj osobné údaje používateľa, ku ktorým poskytol prístup, napríklad jeho telefónne kontakty, poloha, vzdelanie, miesto výkonu práce, rodinný stav a mnohé ďalšie.

V mojom článku rozoberiem systém vyhľadávania a výberu vhodného obsahu pre používateľa podľa jeho preferencií.

\textbf{Keywords:} Odporúčacie systémy, algoritmy, sociálne siete.

\end{abstract}






\section{Úvod}\label{uvod}
Náš svet sa vyvíja každým dňom. Ani systémy odporúčaní (recommendation systems) nestoja na mieste. Obzvlášť zaujímavé je sledovať vývoj odporúčacích systémov sociálnych médií. Vzhľadom na veľký objem informačného toku, ktorý generujú sociálne siete, výskumníci predstavujú rôzne metódy na získavanie relevantných informácií \cite{diego01}. Poskytovanie personalizovaného a adaptívneho obsahu je jednou z najdôležitejších úloh systémov vyhľadávania informácií a odporúčacích systémov \cite{diego01}. K napísaniu tohto článku ma inšpirovalo niekoľko zdrojov. Veľmi ma zaujal článok o odporúčacích systémoch (RS), ktoré využívajú informácie o sociálnych
vzťahoch medzi používateľmi, kde boli sociálne vzťahy rozdelené do troch hlavných výsledkov: dôvera, akcie priateľov a interakcie \cite{diego01}. Narazil som aj na rovnako zaujímavý článok, v ktorom autori skúmali potrebu odporúčacích systémov a zlepšenie výkonu pomocou dodatočných informácií o používateľoch zo sociálnych sietí \cite{chen02}.


Cieľom mojej práce je preskúmať metódy a algoritmy na výber vhodného obsahu pre používateľa v sociálnych sieťach na základe jeho preferencií. Článok sa skladá z piatich častí. Druhá časť  \ref{part2} popisuje dôležitosť odporúčacích systémov. Tretia časť \ref{part3} hovorí o typoch odporúčacích systémov a základných princípoch ich fungovania. Štvrtá časť \ref{part4} popisuje fungovanie odporúčacích systémov na príklade sociálnej siete LinkedIn. Posledná časť \ref{zaver} obsahuje záver.



\section{Dôležitosť odporúčacích systémov}\label{part2}

Systémy odporúčaní sú navrhnuté tak, aby nám pomohli vyhnúť sa zmätku a frustrácii spôsobenej hľadaním požadovanej možnosti v informačne preťaženom systéme\cite{linkedin_pourheidari12}.

Aby sme ukázali dôležitosť odporúčacích systémov, musíme sa pozrieť na množstvo času, ktoré každý deň trávime na sociálnych sieťach.

\begin{figure}[h]
    \centering
    \includegraphics[width=\linewidth]{DataReportal+Digital+2024+Global+Overview+Report+Slide+238.png}
    \caption{Koľko času každý deň trávime na sociálnych sieťach.\newline{} Zdroj:
    \href{https://datareportal.com/reports/digital-2024-deep-dive-the-time-we-spend-on-social-media}{https://datareportal.com/reports}}
    \label{fig:spending_time}
\end{figure}

Na stránke vyššie som našiel informácie o tom, koľko času ľudia strávili na rôznych aplikáciách za jeden mesiac. Pomocou tohto príkladu môžete pochopiť, ako dobre fungujú odporúčacie systémy v sociálnych sieťach, lebo ich úlohou je poskytovať vhodný obsah, aby používateľ trávil na tejto platforme čo najviac času.


Odporúčacie systémy pomáhajú poskytnúť používateľovi presne tie informácie, o ktoré má záujem. Týmto spôsobom sa do značnej miery rieši problém informačného preťaženia\cite{patel04}.




\section{Ako fungujú odporúčacie systémy}\label{part3}

Ako som už spomínal hlavnou úlohou sociálnych sietí je spájať ľudí so spoločnými záujmami. Práve na to potrebujeme odporúčacie systémy.


\begin{table}[h]
\centering
\begin{tabular}{|
>{\columncolor[HTML]{B4C6E7}}l
>{\columncolor[HTML]{B4C6E7}}l |}
\hline
\multicolumn{2}{|c|}{\cellcolor[HTML]{F4B084}\begin{tabular}[c]{@{}c@{}}Odporúčacie   systémy analyzujú akcie používateľa v sociálnej sieti a na zá-\\      klade toho navrhujú:\end{tabular}} \\ \hline
\multicolumn{2}{|l|}{\cellcolor[HTML]{B4C6E7}zaujímavých ľudí} \\ \hline
\multicolumn{2}{|l|}{\cellcolor[HTML]{B4C6E7}vhodný obsah} \\ \hline
\end{tabular}
\label{tab:fungovanie RS}
\caption{Ako fungujú odporúčacie systémy}
\end{table}




V sociálnych sieťach je na poskytovanie presných a vhodných odporúčaní pre používateľov najdôležitejšou vecou systému naučiť sa preferencie používateľov presne z historických informácií o správaní a informácií o sociálnych vzťahoch\cite{li_social_network08}.





\subsection{Výnimočnosť sociálnych sietí}

Výnimočnosť sociálnych sietí spočíva v tom, že okrem spotreby obsahu používateľmi, existujú aj spojenia medzi nimi. To poskytuje ďalšie možnosti lepšieho výberu obsahu pre človeka. Môžeme, napríklad, zvážiť jeden z typov odporúčacích systémov – kolaboratívne filtrovanie.

Collaborative Filtering (CF) je bežne používaná technika v odporúčacích systémoch, tento prístup odporúča používateľovi obsah na základe preferencií iných podobných používateľov \cite{dang03}. Najdôležitejšie je, že tento odporúčací systém nepotrebuje poznať žiadne osobné informácie o človeke. Kolaboratívne filtrovanie pre systémy odporúčaní používa databázu o používateľských preferenciách na predpovedanie ďalších tém alebo produktov, ktoré by sa novému používateľovi mohli páčiť\cite{john_cf06}.



\begin{figure}[h]
    \centering
    \includegraphics[width=0.75\linewidth]{Collab filtering.drawio.png}
    \caption{Kolaboratívne filtrovanie}
    \label{fig:collab filtering}
\end{figure}


Kolaboratívne filtrovanie odhaduje záujem o položku podľa záujmov používateľov „podobných“ cieľovému používateľovi, pričom hodnotenie položky je najpoužívanejšou informáciou v CF\cite{diego01}.

Hlavný rozdiel medzi kolaboratívnym filtrovaním a filtrovaním založeným len na preferenciách používateľa spočíva v tom, že pri kolaboratívnom filtrovaní používateľovi môže byť poskytnutý obsah, ktorý je pre neho nový a neznámy. Deje sa tak práve vďaka prepojeniam medzi používateľmi s podobnými záujmami, ktorí sú následne pridaní do jednej skupiny. Potom sa každému z nich prezentuje podobný alebo úplne nový obsah na základe preferencií ostatných používateľov z tejto skupiny.


\subsection{Hlavné typy odporúčacích systémov a rozdiel medzi nimi}

Okrem kolaboratívneho filtrovania existuje filtrovanie na základe obsahu, hybridné filtrovanie a veľa ďalších široko používaných typov filtrovania RS, ktoré by som chcel podrobnejšie preskúmať nižšie:

\begin{table}[h]
\centering
\scalebox{1.1}{%
\begin{tabular}{
>{\columncolor[HTML]{D9E1F2}}l
>{\columncolor[HTML]{D9E1F2}}l
>{\columncolor[HTML]{D9E1F2}}l
>{\columncolor[HTML]{D9E1F2}}l
>{\columncolor[HTML]{D9E1F2}}l }
\multicolumn{5}{c}{\cellcolor[HTML]{FFD966}Existujú tri hlavné typy filtrovania:} \\
\multicolumn{5}{l}{\cellcolor[HTML]{D9E1F2}1. Kolaboratívne   filtrovanie} \\
\multicolumn{5}{l}{\cellcolor[HTML]{D9E1F2}2. Filtrovanie na   základe obsahu} \\
\multicolumn{5}{l}{\cellcolor[HTML]{D9E1F2}3. Hybridné   filtrovanie}
\end{tabular}%
}
\caption{Hlavné typy odporúčacích systémov}
\label{tab:hlavne_typy_RS}
\end{table}




\begin{itemize}
\item Filtrovanie na základe obsahu (Content-based filtering):

Filtrovanie na základe obsahu je tiež jedným z najdôležitejších typov odporúčacích systémov. Fungovanie tohto typu RS je založené na ponúkaní používateľovi nových videí a publikácií na základe obsahu, ktorý si už prezrel.


\begin{figure}[h]
    \centering
    \includegraphics[width=0.6\linewidth]{content-based.drawio.png}
    \caption{Filtrovanie na základe obsahu}
    \label{fig:content-based}
\end{figure}


Hlavný rozdiel medzi CF a systémami odporúčaní založenými na obsahu je v tom, že CF používa iba údaje o hodnotení používateľských položiek na vytváranie predpovedí a odporúčaní, zatiaľ čo systémy odporúčaní založené na obsahu sa spoliehajú na funkcie používateľov a položky pre predpovede\cite{survey_cf05}. Content-based filtering vyberá obsah na základe už prezretého obsahu a preferencií používateľa. Najlepšie zodpovedajúce položky sa odporúčajú porovnaním rôznych kandidátskych položiek s položkami, ktoré používateľ predtým hodnotil\cite{thorat_cont-b07}. Ale Collaborative Filtering vyberá obsah na základe preferencií ľudí s podobnými záujmami.


\item Hybridné filtrovanie (Hybrid filtering):

Veľké spoločnosti, ktoré vytvárajú a zlepšujú systémy odporúčaní pre svoje sociálne siete, v súčasnosti nepoužívajú iba jeden typ RS, pretože ich úlohou je čo najpresnejšie vybrať obsah pre používateľa. Na tento účel ľudia vyvinuli hybridné systémy odporúčaní, ktoré kombinujú niekoľko typov RS. Väčšinou tento typ RS kombinuje Kolaboratívne filtrovanie a filtrovanie na základe obsahu. Kombinácia oboch typov vedie k zvýšeniu všeobecných vedomostí, čo prispieva k lepším odporúčaniam\cite{geetha_hybrid09}. Hybridné filtrovanie kombinuje spojenia medzi používateľmi a informácie o osobných preferenciách každého používateľa.


\begin{figure}[h]
    \centering
    \includegraphics[width=0.6\linewidth]{hybrid_filtering.drawio.png}
    \caption{Hybridné filtrovanie}
    \label{fig:hybrid-filtering}
\end{figure}



\item Social Network-Based recommendations:

Tento typ RS je tiež hybridný a založený na spojení medzi priateľmi na sociálnych sieťach. Na rozdiel od tradičných systémov odporúčaní by systémy odporúčaní v sociálnej sieti mohli využívať aj online sociálne vzťahy medzi používateľmi (t. j. priateľstvo) na zlepšenie výkonu odporúčaní\cite{li_social_network08}. Ak sú používatelia priatelia a sledujú sa navzájom, tak musia mať spoločné záujmy. Toto je hlavná myšlienka odporúčaní založených na sociálnej sieti. Tento typ systému odporúčaní analyzuje preferencie priateľov používateľa v tejto sieti a na základe toho navrhuje profily iných ľudí a videá, ktoré jeho priatelia preferujú.

\begin{figure}[h]
    \centering
    \includegraphics[width=0.75\linewidth]{social_network_based.drawio.png}
    \caption{Social Network-Based recommendations}
    \label{fig:network-based}
\end{figure}

Obsah, ktorý bude používateľovi poskytnutý, závisí nielen od akcií samotného používateľa v sieti, ale aj od preferencií jeho priateľov.


\item Porovnanie a výsledky:

\begin{figure}[h]
    \centering
    \includegraphics[width=0.6\linewidth]{comparing.drawio.png}
    \caption{Porovnanie všetkých vyššie uvedených typov filtrácie}
    \label{fig:comparing}
\end{figure}

Hybridné filtrovanie je najefektívnejšie, pretože kombinuje kolaboratívne filtrovanie a filtrovanie na základe obsahu, čo umožňuje presnejšie poskytnúť obsah používateľovi. Kolaboratívne filtrovanie je efektívnejšie ako filtrovanie  na základe obsahu, pretože tento typ systému odporúčaní pridáva používateľa do skupiny používateľov s podobnými záujmami a navrhuje obsah, ktorý sa páčil iným v tejto skupine. Tento typ RS je založený na spojeniach medzi používateľmi a umožňuje poskytovať úplne nový obsah. Toto je výhoda tohto typu oproti filtrovaniu na základe obsahu, pretože filtrovanie na základe obsahu je založené iba na preferenciách používateľa a neponúka nič nové.

Treba tiež povedať, že veľkou nevýhodou všetkých uvedených typov odporúčacích systémov je problém studeného štartu (cold start). Problém so studeným štartom súvisí s nedostatkom informácií dostupných v algoritme odporúčaní\cite{lika_cold_strart10}. Studený štart je stav, keď odporúčací systém ešte nevie nič o používateľovi a jeho preferenciách. Stáva sa to vo fáze registrácie na konkrétnej sociálnej sieti, keď používateľ ešte nezačal sledovať určitý obsah.


\end{itemize}




\section{Fungovanie odporúčacích systémov na príklade LinkedIn}\label{part4}

V tejto časti by som chcel preskúmať použitie odporúčacích systémov na konkrétnom príklade. Rozhodol som sa hovoriť o tom, ako fungujú odporúčacie systémy na príklade sociálnej siete LinkedIn. Zaujala ma práve táto sociálna sieť, pretože má celkom zaujímavý algoritmus poskytovania obsahu používateľovi, ktorý sa veľmi líši od ostatných sociálnych sietí.



LinkedIn je sociálna sieť na rozšírenie pracovných kontaktov, kde môžu používatelia z celého sveta zverejniť svoj profil a môžu sledovať profily iných používateľov. Táto sociálna sieť sa využíva na prácu, takže profil človeka je jeho životopis, kde uvádza svoje vzdelanie, pracovné skúsenosti, zručnosti atď. Používatelia tu môžu zdieľať svoje názory a zanechať recenzie o práci v rôznych spoločnostiach.


Pracovný ekosystém LinkedIn bol navrhnutý ako platforma na spájanie poskytovateľov a uchádzačov o zamestnanie a ako trh na efektívne spájanie potenciálnych kandidátov s voľnými pracovnými miestami\cite{linkedin_recruit13}. Veľké percento recruiterov využíva LinkedIn na hľadanie kandidátov\cite{linkedin11}. Preto je v LinkedIn veľmi dôležité správne fungovanie odporúčacích systémov, čo uľahčuje vyhľadávanie ľudí a spoločností z podobnej oblasti.


Na rozdiel od iných sociálnych sietí, odporúčacie systémy LinkedIn nemajú taký studený štart. Je to spôsobené tým, že používateľ pri registrácii zadáva osobné údaje o sebe (zostavuje životopis). V tejto fáze už RS môžu začať odporúčať nových ľudí a spoločností, o ktoré môže mať používateľ záujem, napriek tomu, že zatiaľ nikoho nesleduje.


LinkedIn možno veľmi dobre použiť ako príklad na demonštráciu fungovania Social Network-Based recommendations, pretože odporúčania v LinkedIn priamo závisia od spojení medzi používateľmi. Na rozdiel od iných sociálnych sietí nemôžete zdieľať príspevok s obsahom na akúkoľvek tému. Pred uverejnením príspevku ho najprv skontroluje samotný filtračný systém LinkedIn a rozhodne, či ide o spam alebo kvalitu, pretože táto sieť je určená pre ľudí, ktorí hľadajú užitočné informácie týkajúce sa práce v ich oblasti, nie obsah na pobavenie. V ďalšom kroku systém filtrovania LinkedIn rozošle príspevok používateľa jeho sledovateľom a zistí, či sú pre nich informácie užitočné a ako ľudia z tejto oblasti reagujú na tento príspevok. Ak používatelia zareagujú pozitívne, v ďalšom kroku filtrovací systém odošle príspevok ich sledovateľom, ktorí by mohli mať záujem o tento obsah. Systém využíva osobný profil používateľa na zistenie jeho preferencií a rozhodne, či mu pošle tento príspevok alebo nie. Systém filtrovania tiež určuje záujmy používateľa na základe jeho predchádzajúcich aktivít, aby mohol posielať obsah od používateľov, s ktorými najčastejšie komunikuje.




\section{Záver}\label{zaver}

Na sociálnych sieťach k dispozícii je veľa rôznych informácií. Potrebujeme odporúčacie systémy, ktoré ich roztriedia a zobrazia nie všetko, ale len to, čo nás zaujíma, aby sme nemuseli tráviť veľa času hľadaním vecí, ktoré sa nám páčia. Na tento účel existujú tri hlavné typy filtrovania, ktoré sa používajú v rôznych sociálnych sieťach: kolaboratívne filtrovanie, filtrovanie na základe obsahu a hybridné filtrovanie. Z porovnania všetkých typov vyplýva, že hybridné filtrovanie je najefektívnejšie, pretože kombinuje aj kolaboratívne filtrovanie aj filtrovanie na základe obsahu, čo umožňuje presnejšie poskytnúť obsah používateľovi. Dobrým príkladom využitia hybridného typu filtrovania Social Network-Based recommendations je sociálna sieť LinkedIn, lebo odporúčania v LinkedIn priamo závisia od spojení medzi používateľmi, nielen od vlastných preferencií.

Odporúčacie systémy sociálnych médií sú v súčasnosti veľmi dôležité. Pracujúcim ľuďom pomáhajú nájsť užitočné informácie pre ich prácu a ľuďom, ktorí chcú si užiť zaujímavý čas, pomáhajú nájsť obsah, ktorý sa im páči.


\bibliography{references}
\bibliographystyle{plain}

\end{document}
