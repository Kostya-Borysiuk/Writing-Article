% Metódy inžinierskej práce

\documentclass[10pt,twoside,slovak,a4paper]{article}

\usepackage[slovak]{babel}
%\usepackage[T1]{fontenc}
\usepackage[IL2]{fontenc} % lepšia sadzba písmena Ľ než v T1
\usepackage[utf8]{inputenc}
\usepackage{graphicx}
\usepackage{url} % príkaz \url na formátovanie URL
\usepackage{hyperref} % odkazy v texte budú aktívne (pri niektorých triedach dokumentov spôsobuje posun textu)

\usepackage{cite}
%\usepackage{times}


\title{Výber zaujímavých ľudí a obsahu pre používateľa na sociálnych sieťach\thanks{Semestrálny projekt v predmete Metódy inžinierskej práce, ak. rok 2024/25, vedenie: Richard Marko}} % meno a priezvisko vyučujúceho na cvičeniach

\author{Kostiantyn Borysiuk\\[2pt]
	{\small Slovenská technická univerzita v Bratislave}\\
	{\small Fakulta informatiky a informačných technológií}\\
	{\small \texttt{xborysiuk@stuba.sk}}
	}

\date{\small 3. októbra 2024} % upravte



\begin{document}

\maketitle

\begin{abstract}
V modernom svete takmer každý človek, ktorý má prístup na internet, používa jednu alebo druhú sociálnu sieť, vďaka čomu je moja téma veľmi dôležitá. Hlavnou úlohou sociálnych sietí je spájať ľudí so spoločnými záujmami.
Odporúčacie systémy analyzujú akcie používateľa v sieti a na základe jeho záujmov, prezeraných materiálov, ľudí, ktorých sleduje atď., navrhujú nový odporúčaný obsah. Analyzujú sa aj osobné údaje používateľa, ku ktorým poskytol prístup, napríklad jeho telefónne kontakty, poloha, vzdelanie, miesto výkonu práce, rodinný stav a mnohé ďalšie.

V mojom článku rozoberiem systém vyhľadávania a výberu vhodného obsahu pre používateľa podľa jeho preferencií.
\end{abstract}



\section{Úvod}
Náš svet sa vyvíja každým dňom. Ani systémy odporúčaní nestoja na mieste. Obzvlášť zaujímavé je sledovať vývoj odporúčacích systémov sociálnych médií. Vzhľadom na veľký objem informačného toku, ktorý generujú sociálne siete, výskumníci predstavujú rôzne metódy na získavanie relevantných informácií \cite{diego01}. Poskytovanie personalizovaného a adaptívneho obsahu je výskumným problémom v oblasti vyhľadávania informácií (IR) a odporúčacích systémov (RS) \cite{diego01}. K napísaniu tohto článku ma inšpirovalo niekoľko zdrojov. Veľmi ma zaujal článok o odporúčacích systémoch, ktoré využívajú informácie o sociálnych
vzťahoch medzi používateľmi, kde boli sociálne vzťahy rozdelené do troch hlavných výsledkov: dôvera, akcie priateľov a interakcie \cite{diego01}. Narazil som aj na rovnako zaujímavý článok, v ktorom autori skúmali potrebu odporúčacích systémov a zlepšenie výkonu pomocou dodatočných informácií o používateľoch zo sociálnych sietí \cite{chen02}.

Cieľom mojej práce je preskúmať metódy a algoritmy na výber vhodného obsahu pre používateľa v sociálnych sieťach na základe jeho preferencií.









\section{Dôležitosť dporúčacích systémov}

Aby sme ukázali dôležitosť odporúčacích systémov, musíme sa pozrieť na množstvo času, ktoré každý deň strávime na sociálnych sieťach.

\begin{figure}[h]
    \centering
    \includegraphics[width=\linewidth]{DataReportal+Digital+2024+Global+Overview+Report+Slide+238.png}
    \caption{Zdroj:\url{https://datareportal.com/reports/digital-2024-deep-dive-the-time-we-spend-on-social-media}}
    \label{fig:enter-label}
\end{figure}

Na stránke vyššie som našiel informácie o tom, koľko času ľudia strávili na rôznych aplikáciách za jeden mesiac. Pomocou tohto príkladu môžete pochopiť, ako dobre fungujú odporúčacie systémy v sociálnych sieťach.








\section{Ako fungujú odporúčacie systémy}

Ako som už spomínal hlavnou úlohou sociálnych sietí je spájať ľudí so spoločnými záujmami. Práve na to potrebujeme odporúčacie systémy.

Odporúčacie systémy analyzujú akcie používateľa v sociálnej sieti a na základe toho navrhujú:

\begin{itemize}
\item zaujímavých ľudí
\item vhodný obsah
\end{itemize}





\subsection{Výnimočnosť sociálnych sietí}

Výnimočnosť sociálnych sietí spočíva v tom, že okrem spotreby obsahu používateľmi, existujú aj spojenia medzi nimi. To poskytuje ďalšie možnosti lepšieho výberu obsahu pre človeka. Môžeme, napríklad, zvážiť jeden z typov odporúčacích systémov – Kolaboratívne filtrovanie.

Collaborative Filtering (CF) je bežne používaná technika v odporúčacích systémoch, tento prístup odporúča používateľovi obsah na základe preferencií iných podobných používateľov.\cite{dang03}
Najdôležitejšie je, že tento odporúčací systém nepotrebuje poznať žiadne osobné informácie o človeke.



\subsection{Na základe čoho sa vyberá obsah}

Odporúčaný obsah sa poskytuje nie len na základe činností používateľa na sociálnej sieti a jeho osobných údajov.

Okrem kolaboratívneho filtrovania existuje ešte veľa rôznych typov odporúčacích systémov:

\begin{itemize}
\item Filtrovanie na základe obsahu: Navrhovanie nového obsahu a publikácií používateľovi na základe obsahu, ktorý si už prezrel.
\item Social Network-Based recommendations: Analýza ľudí, ktorých sledujú priatelia používateľa, a odporúčanie mu ich.
\item Post and Content Recommendations: Tieto odporúčania sú založené na interakcii používateľa s príspevkami. Na základe reakcií používateľov na príspevky sa navrhujú nové príspevky daného charakteru.
\end{itemize}



\section{Záver} \label{zaver}

Na sociálnych sieťach k dispozícii je veľa rôznych informácií. Potrebujeme odporúčacie systémy, ktoré ich roztriedia a zobrazia nie všetko, ale len to, čo nás zaujíma, aby sme nemuseli tráviť veľa času hľadaním vecí, ktoré sa nám páčia.


\bibliography{references}
\bibliographystyle{plain}

\end{document}
